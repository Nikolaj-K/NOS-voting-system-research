\documentclass[12pt, a4paper, twoside, titlepage]{article}
% font size could be 10pt (default), 11pt or 12 pt
% paper size coulde be letterpaper (default), legalpaper, executivepaper,
% a4paper, a5paper or b5paper
% side coulde be oneside (default) or twoside 
% columns coulde be onecolumn (default) or twocolumn
% graphics coulde be final (default) or draft 
%
% titlepage coulde be notitlepage (default) or titlepage which 
% makes an extra page for title 
% 
% paper alignment coulde be portrait (default) or landscape 
%
% equations coulde be 
%   default number of the equation on the rigth and equation centered 
%   leqno number on the left and equation centered 
%   fleqn number on the rigth and  equation on the left side
%	
\title{A not so small \LaTeX{} Article Template\thanks{To your mother}}
\author{Your Name  \\
	Your Company / University  \\
	\and 
	The Other Dude \\
	His Company / University \\
	}

\date{\today} 
% \date{\today} date coulde be today 
% \date{25.12.00} or be a certain date
% \date{ } or there is no date 
\begin{document}
% Hint: \title{what ever}, \author{who care} and \date{when ever} could stand 
% before or after the \begin{document} command 
% BUT the \maketitle command MUST come AFTER the \begin{document} command! 
\maketitle


\begin{abstract}
Short introduction to subject of the paper \ldots 
\end{abstract}

%\tableofcontents % create a table of contens 


\section{Ranking system for a dApp store}
blablaisieren 


\section{Decentralized ranking systems in context}
\subsection{Motivation} 
In this paper, we advance the discussion of voting systems on distributed ledgers such as blockchains. The core benefit of such frameworks is that that vote-casting and voting-result-computation is perfectly transparent and thus always audible.
More concretely, we will not study the consensus protocols enabling the decentralized voting system themselves, but instead review and propose\\
$\bullet$ algorithms for poll evaluations resulting in a sorted lists\\
$\bullet$ innovations made possible through the underlying audible framework

The former can manifest as a mechanisms for the evaluation of opinions invoked by the voters, which, taken together, result in a average ranking. One of our main motivations here is the design of a voting system for a dApps store. So while the subject of the poll algorithms discussed may be anything from politicians to fruits, we will consider the rating of software applications. The vote casting actors will thus be referred to as `users` and correspond to whitelisted addresses on the decentralized ledger. The whitelisting itself need not be result of a decentralized process. 

In this context, the most notable innovation that comes with audible polling is the verifiable claim for rewards based of ``on-chain`` actions. For each poll or voting round, the final result can be put in context with the individual user's vote and thus used to compute a transparent reward claim tied to the user address.

\subsection{Companies that use voting systems}

\subsection{Requirements for NOS in comparison to what other companies use}
what's different here 

a) blockchain consensus 

b) dApp rankings 

c) Gain from being votes high up.

\section{Scientific view}
\subsection{properties of voting systems (viewed on their own)} % wie im Satz
\subsection{rule out and classify for the means of a) and b)}

\subsection{existing voting systems}
\subsection{differentiation; part of upper subsection}

\section{economical perspective} %?????????????????????????????

\section{how the properties apply for other companies and how steemit is bad etc}

\section{propose an algorithm}





blabla2













\paragraph{Outline}
First we start with a little example of the article class, which is an 
important documentclass. But there would be other documentclasses like 
book \ref{book}, report \ref{report} and letter \ref{letter} which are 
described in Section \ref{documentclasses}. Finally, Section 
\ref{conclusions} gives the conclusions.



\section{Documentclasses} \label{documentclasses}

\begin{itemize}
\item article
\item book 
\item report 
\item letter 
\end{itemize}


\begin{enumerate}
\item article
\item book 
\item report 
\item letter 
\end{enumerate}

\begin{description}
\item[article\label{article}]{Article is \ldots}
\item[book\label{book}]{The book class \ldots}
\item[report\label{report}]{Report gives you \ldots}
\item[letter\label{letter}]{If you want to write a letter.}
\end{description}

\section{tabular}
No paper without a tabular!

\begin{tabular}{|l|c|r|p{2cm}|}
\hline
first column & second column & third column & fourth column \\
\hline 
l stand for left & c for center & r for right & and p for predefined size \\
\hline 
\end{tabular} 


\section{some math}
Math in text is called in line math just put \$ character around 
the math think. Like $ a^2 + b^2 = c^2 $. It looks better if you use 
this 
\[a^2 + b^2 = c^2\]

\section{Conclusions}\label{conclusions}
There is no longer \LaTeX{} example which was written by \cite{doe}.

\begin{thebibliography}{9}
\bibitem[Doe]{doe} \emph{First and last \LaTeX{} example.},
John Doe 50 B.C. 
\end{thebibliography}

\end{document}