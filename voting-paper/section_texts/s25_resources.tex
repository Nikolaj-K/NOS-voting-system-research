
\subsection{Paper overview}

In the remainder of the Introduction section, we shine light on the most relevant ones for our current context or the background we make use of.

(... yadda yadda)

The paper closes with an extensive literature list and links. [LINK ]

\subsection{Resources} 

\subsubsection{The mathematics of voting and elections: A hands-on approach.}
The ``Handbook of Electoral System Choice'' by Josep and Colomer covers the selection of voting systems, but that treatie is about settling the choice of political electoral system to determine parties from and for countries. In contrast, highly voted apps benefit from exposure but are not voted into the status of rule makers.
Furthermore, we don't have winners as such, but instead obtain exposure.

\subsubsection{Electoral Knowledge Network}
This website provides information and customised advice on electoral processes. 
The ``Administration and Cost of Elections'' (short: "ACE") - Projects promotes electoral processes the project's team considers credible and transparent. 
Besides other information, the website contains global statistics and data and an Encyclopaedia of Elections which covers topics in elections management. \\
$\bullet$ \href{http://www.aceproject.org/}{aceproject.org}\\
$\bullet$ \href{http://aceproject.org/ace-en}{aceproject.org/ace-en}

\subsubsection{Electorama Wiki} 
$\bullet$ \href{https://wiki.electorama.com}{Main page}\\
$\bullet$ \href{hhttps://wiki.electorama.com/wiki/Special:AllPages}{All pages}

\subsubsection{---} 
{todo: Why voting dApps isn't voting parties}
{ordered result}


\href{https://en.wikipedia.org/wiki/Types_of_democracy}{not going into the direct/representative debate (Athen/Roman)}

\subsubsection{---} 

Real-World-Electronic-Voting-Analysis-Deployment
\href{https://www.amazon.com/Real-World-Electronic-Voting-Analysis-Deployment-ebook/dp/B01N6B7DGW}