
\section{Design for a ranking dApp}
\subsection{Desired voting properties}
We now want to reflect upon the notions discussed in the preceeding section ..

(todo: this might be a bit intermingled with the above one atm.)

{\color{blue}{
\begin{verbatim}


4) Then rule out and classify for the means of a) and b)

5) Also point out how those apply for other companies. TODO

## Basic summary 

### Considerations we want and what we don't want

The voting system should fulfill the following criteria: 

1) User's values: Reputation, Voting Power (reputation and voting power should be dependent, reputation = voting power possible)

2) Reputation grows if user's vote is in consensus with community's votes 
   * a) various ways to predefine what "consensus" does mean
   idea: determine consensus in periods of a week 
   * b)!!! but users should not be able to vote for apps they haven't used only to get rewarded and users should not be motivated to vote for apps only because they already have a high number of votes and therefore are sure to 
   be in consensus
   idea: reward could be higher for the first voters
   idea: votes should cost
   
3) Reputation decreases if user's vote isn't in consenus with communitys's votes

4) bad apps may be reported by users (= some sort of downvoting in very, very bad cases) 

5) the reputation of users should sink whenever their programmed apps get reported

6) if an app has a large number of reports (limit to be predetermined), users should get a warning before downloading and/or running the app

7) apps' values: number of votes, number of reports;
* evaluation based on: 
- a) usage rate (!!! but apps that are needed more often shouldn't have an advantage over apps that are needed rarely)
- b) number of votes
- c) number of reports
- d) time that an app has been in the market
- e) extent of NOS user base

8) user's reward for a consensus vote should not be dependent from user's voting power or reputation (negative model: steemit)

9) reputation and voting power might be limited 
    - a) idea: including a parameter so that at a very high level the increment of reputation converges to 0 

10) data onchain/offchain? 

11) calculation costs

12) for a listed ranking, a dApp should only be added to the average if it has a certain number of votes. This number can be a pecentage of the user base, but shouldn't grow too high (so that new dApps can be added even if the user base is huge)
 
?) Rewards for dApp producers? (If so, it should probably not depend on ranking.)

########################################################

## Classification

### Terminology and fundamental notions 
###### ! differentiate between voting systems and properties of such.
	
want <Ranking>
 Majority (=absolute majority) => Plurality
but want people to not cast only one vote (otherwise we can too few votes).

how to count votes? Especially since there are several votes per person, i.e. up to
|users| * |dApps|
votes

######  *) Ranking
==> sign of trust and value
==> dApp producer gets attention

will argue for a form of
https://en.wikipedia.org/wiki/Cardinal_voting
TODO: look at all of those and work out pros and cons and differences.

######  *) Anonymity, Neutrality, Monotonicity
Most fundamental base voting system is majority rule characterized by: 
	- Anonymity (Hodge p.4)
	- Neutrality
	- Monotonicity
	
Facebook likes

- Monotonicity is almost self-evidently valuable. 
- Neutrality is a free market rule. (While nOS has power over the system and can form coalitaitons with individual dApp designers, there is likely no from nOS relizing any sort of bonus in the ranking system. Of course, nOS might promote particular dApps independent of the ranking) 
- Anonymity ... (see Steemit and Anonymity (reputation)
	=> aggregation problem
	=> solition via brakets of last time period
	exposure effect
	
Regading Steemit ... keep an eye on rewards: How it's solved by steemit, complexity, problems.


######  *) Quotas
This is just details: 

In our case, 

1) Quotas <=> Could in principle be used as a cap for when a dApp even enters the ranking. 
On the lower end (i.e. having a low quota) => Could be unfair w.r.t. exposure differences. Being ranked low is enough of a punishment for that.
One could imagine a quota at the ghier end (extra bonus exposue for breaking a benchmark). This can be done with relative numbers, e.g. if the dApps gain a significant fraction of the total votes (see also Required Voting below) or a significat fraction relative to the votes for other dApps (e.g. whenever one dApp gains 110% the number of votes of the dApps below them).

2) Given a computed ranking, quotas <=> Could be used for a hard cap of which apps aren't show on the website

Can be set for when typical numbers (user base, dApp base, size) are established. Such a quote could depend on the website layout or be a running number depending on the number of dApps.
We have a ongoing running voting process with varying user base size and number of dApps.

The majority of dApps should of couse be accessible and searchable on the platform even if they are low in the rank
=> keep an eye on rewards: Steemit system, complications

Both of the above => flagging


######  *) Required voting
Note that majority rule doens't require everybody to vote and this is probably also not something we want to enforce (note on GAS being rewards only when voted). 
This (the lack of fixed number of total votes upfront) makes the design of the decission algorithm harder as it makes benchmarks more volatile, so that previous voting rounds lose predictive content.

######  *) Ties
Due to the large numbers of dApps to be voted for and the high number of voting rounds (and thus low stakes per voting rounds), ties are no real concern for the platform.

######  *) Breaking Neutrality.
While it's not a crucial featues, a tie can easily be deterministically resolved by reflection on timely day, e.g. sign up date (e.g. older or more recent dApps have a bonus) or previous ranking results (e.g. previous winners or non-winners have a bonus). Note that here the finaly ranking would be influenced by the last one.
What makes Neutrality for us different than e.g. for political elections considered in isolation is that we can have a continuous stream of new dApps .
It raises the question of how self-contained a voting round should be in general. An argument for not breaking neutrality (beyond deciding ties) is that it makes process computationally more compact.

One might be tempted to say it also frees us from incoorporating previous data at all. However, below when talking about gamification aspects, which is tied to long term participation and (informal) platform reputation, we'll argue why the past shoud however indeed be involved to compute reward


###### For our consideration

* Reputation ...
* Voting Power ...
* Flag ... dApps can be annotated/singled out for being suspicious/spam 

##### Established voting mechanisms


\end{verbatim}
}}%% end color




\subsubsection{On exposure/money as reward}

{\color{blue}{
\begin{verbatim}


What does exposure mean?
	* a spot in a list (as opposed to relative quantitative gain, as in "Proportional representation")

Interested in all dApps => We want to use ranking

Positive votes (we vote who we want, not who we don't want). The ranking implicitly provites a mechanism to establish who's at the bottom of the food chain (see flagging) 

#### For the users
* Rewards
* Reputation?


\end{verbatim}
}}%% end color





