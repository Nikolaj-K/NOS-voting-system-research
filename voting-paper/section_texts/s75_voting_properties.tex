
\section{On voting}
We want to tap into the rich pool of established knowledge on voting voting systems, to inform our options and guide our decisions in designing suitable algorithms. 

{\color{blue}{
\begin{verbatim}

3) We're first going to give an overview of 
	* properties of voting systems (viewed on their own)
	* (differentiation between) popular existing voting systems. 
	
\end{verbatim}
}}%% end color

\subsection{On existing voting systems}

%TODO: specify, for which elections (2 candidates etc) the following holds
In this section we recall the basic established definitions and then reflect upon our needs.
Commonly, the three properties that a voting system should fulfil in order to be fair are anonymity, neutrality and monotonicity. For the elections of one of two possible candidates, these properties are formalised as follows: \\

$\bullet$ {\emph{Anonymity}\footnote{This notion of anonymity is unrelated to the anonymity of a user using an app.}}:
A voting system is said to be anonymous if it treats all voters equally. In other words, if any two voters trade ballots, this shouldn't change the election's outcome. \\
Concerning a dApp-voting-system, it is up for discussion in what form a property akin to this should be implemented. For example, it could be sensible to give users who have a high reputation, which indicates their knowledge, or users who hold a large stake and therefore are likely to want the best for the platform, more voting power than others. Certainly it would establish an unwanted great inequality between users if the relation between voting power and reputation or stake was a linear one. We discuss this topic further in 
%\ref{label} 
%S. 19: Majority rule is the voting system that elects the candidate who receives more than half of the votes, if such a candidate exists. If there is no such candidate, then majority rule results in a tie, with no winner elected. 

$\bullet$ {\emph{Neutrality}}: A voting system is said to be neutral if it treats all candidates equally. This means that if every voter switched their vote from one candidate to another, the outcome should change accordingly. \\
This point is not about the voters. For dApp rating, the neutrality property is about whether the voting-evaluation algorithm distinguishes all the dApps that are applicable to partake in the voting round. 

$\bullet$ {\emph{Monotonicity}}: A voting system is said to be monotone if it is impossible for a candidate to change from winning to losing by gaining additional votes and to change from losing to winning by losing votes without gaining others. \\

The so-called May's Theorem states that majority rule %TODO: see \ref{} 
is the only voting system for two-candidate-elections that is anonymous, neutral, and monotone, and that avoids the possibility of ties. However, majority rule is no option for the dApp-voting-system because there will be more than two candidates and anonymity is likely an undesired feature. \\

%\cite{Handsonapproach} The Mathematics of Voting and Elections: A Hands-On Approach. Chapter One; maybe TODO: quota systems

One class of methods, of which the majority rule can be viewed as a special case, is the plurality rule. It is the voting system that elects the candidate who receives the largest number of votes, even if that number is less than half of the total number of votes cast. Here, only when two or more candidates receive the same number of votes (and more than the number of votes received by any of the other candidate), the plurality method does result in a tie. \\
Concerning an election for dApp-ranking, the possibility of ties is no downside as there is no need to determine a single winner.\\ %S. 19
Rather, what is needed is a system that leads to some sort of preference order of all dApps. Such preference order produced by the voting system is called {\emph{societal preference order}} since it can be thought of as the ranking of the candidates that, according to the voting system being used, best reflects the voters' will. \\
%n protip: avoid " in TeX
There are various system that can be used to determine the societal preference order. One property of such systems that might sound sensible at first, is the following: \\
A voting system is said to satisfy the {\emph{majority criterion}} if whenever a candidate is ranked first by a majority of the voters, that candidate will also be ranked first in the corresponding societal preference order. Below we give an example for why this would be no legitimate property for the dApp-voting-system.\\
A voting system that does not fulfil the majority criterion is the so-called {\emph{Borda count}}, which uses a non-trivial {\emph{point system}} to determine the overall rankings and is often used in collegiate sports polls, for example. In an election with $n$ candidates it works as follows: \\
Firstly, each voter submits a ballot that contains his or her individual preference order of all the candidates. 
Then points are awarded to each candidate for each ballot cast, according to the following rule: 
A $m$-place ranking is worth $n-m$ points (where $1\leq m \leq n$). In other words, a first-place ranking is worth $n-1$ points, a second-place ranking is worth $n-2$ points, and so on. 
Finally the candidate whose total number of points from all of the ballots is the largest is declared the winner and the corresponding societal preference order is determined by the number of points each candidate has got from largest to smallest. If there is more than one candidate with the largest number of points, a tie occurs. Also some sort of tie occurs in the societal preference order whenever candidates receive the same number of total points. These candidates then occupy consecutive indistinguishable positions in the preference order. \\
Intuitively, Borda count also appears to be quite fair. However, it violates the majority criterion. For the purpose of the dApp-voting-system we consider using some sort of variation of the Borda count more appropriate than demand the majority criterion. Imagine an extreme situation where a small majority of voters (users) rank a specific dApp at the top position place, but all of the other voters rank it last on their personal preference. Obviously, it would make no sense to declare this dApp the winner and rank it first in a societal preference order. Determining its rank according to the Borda count seems much more legitimate. Thus, a dApp-voting-system need not fulfil the majority criterion and will not use the plurality method, but rather use a variant of Borda count to guarantee a sensible societal preference order. Taking account of personal and societal preference orders, for elections with more than two candidates, the three properties of fair voting systems have to be redefined: \\
%%n is this quite different than before? why list all again?
$\bullet$ Anonymity: A voting system is said to be anonymous if it treats all voters equally. This means that if any two voters traded their personal preference orders, the outcome of the resulting societal preference order should not change. \\
As mentioned before, this is no property we want the dApp-voting-system to fulfil, see \\ %\ref{}. 
$\bullet$ Neutrality: A voting system is said to be neutral if it treats all candidates equally. This means that if every voter switched the positions of two specific candidates on their personal preference orders, the positions of these two candidates in the resulting societal preference order would be switched accordingly. \\
$\bullet$ Monotonicity: A voting system is said to be monotone if it is impossible for a candidate to go from winning to losing or to experience a decrease in rank on the resulting societal preference order whenever changes in favour of that candidate, but no changes in disadvantage of that candidate, occur on individual preference ballots. \\
%großes Buch wo alles drinsteht, Chapter Two

A candidate who would defeat every other candidate in a head-to-head election (under majority rule) is called a {\emph{Condorcet winner}}. If a candidate would lose to every other candidate in a head-to-head contest (under majority rule) is called a {\emph{Condorcet loser}}. Two properties one clearly would want a voting system to fulfil are the following: 
%%n those are not properties but instead definitons. ??
\begin{enumerate}
\item A voting system is said to satisfy the {\emph{Condorcet winner criterion (CWC)}}, if it always elects the Condorcet winner whenever one exists.
\item A voting system is said to satisfy the {\emph{Condorcet loser criterion (CLC)}}, if it always elects the Condorcet loser whenever one exists. 
\end{enumerate}
These definitions involve two-candidate-elections. It can be challenging to design a voting system with more than two candidates in a way that it fulfils a Condorcet criterion. The Borda count, for example, do not (as it violates majority rule). \\
%But since the Condorcet properties clearly are valuable requirements concerning a voting system, the dApp-store-voting system should fulfil them. The so-called sequential pairwise voting, which is explained in the following lines, does. A voting system we propose below, a variation of this system, turns out to do so, too.\\ %später zeigen, dass es das Kriterium erfüllt
%Sequential pairwise voting: \\
%To determine the winner and the societal preference order according to sequential pairwise voting, at first, the voters have to choose between two candidates. Majority rule is used to decide the winner. Next, the voters have to choose between the winner of the preceding contest and the next candidate. Then they have to choose between the winner of the latter contest and the next candidate and so on, whereby the winner always is decided according to majority rule. Finally, the winner from the last two-candidate-contest is declared the winner of the whole election. \\
% Though sequential pairwise voting has the benefit of always electing a Condorcet winner if one exists, it comes with great downsides. First of all, it would produce a societal preference order that doesn't have the mathematical property of transitivity and therefore would not be sensible. Furthermore, in sequential pairwise voting, an order in which new candidates are to be introduced into the head-to-head comparisons has to be specified beforehand. In elections where no Condorcet winner exists, exactly this order, called agenda, has an enormous influence on who wins the election. In other words, if there is no Condorcet winner, this system is highly manipulable and violates the fundamental property of neutrality, which is why sequential pairwise voting clearly can't be said to be a fair method. \\
Another criterion that can be considered related to the assessment of the utility and sensibility of a voting system is the so-called "Independence of Irrelevant Alternatives" - criterion, defined as follows: \\
A voting system is said to satisfy the irrelevant alternatives criterion (IIA) if the societal preference between any two candidates is only dependent on the individual voters' preferences between those two candidates (and not affected by the candidacy of any other candidate). \\
However, none of the voting systems discussed so far satisfies this criterion. \\
In 1951, Kenneth arrow in the context of his Ph.D.-thesis published the commonly called "Arrow's impossibility theorem" which states that no voting system thathas the five desired qualities mentioned above does exist. In 1972, this discovery even earned him the Nobel Prize in economic science. Various versions of the theorem exist, but what it states, more precisely, is the following: \\
You can consider a voting system to be a rule that assigns to each possible collection of preference orders for a poll some kind of societal preference order, or more concretely, \\
we define it as a function that takes as input a collection of transitive preference orders of all the voters and produces as output a transitive societal preference order that represents the will of the electorate. \\
This definition doesn't rule out the possibility of ties, neither ties in individual preference ballots nor ties in the resulting societal preference orders. So if one defines the societal preference order of pairwise sequential voting as the winner followed by a tie of all the other candidates, even pairwise sequential voting fits in with that defintion of a voting system. \\
The mathematical conditions Arrow had in mind can be interpreted as follows: 
\begin{enumerate} % from the voting book, page 68; andere Kriterien auf Wikipedia: https://en.wikipedia.org/wiki/Kenneth_Arrow, https://en.wikipedia.org/wiki/Arrow%27s_impossibility_theorem, https://plato.stanford.edu/entries/arrows-theorem/, 
%TODO: Arrow's theorem http://pi.math.cornell.edu/~mec/Summer2008/anema/maystheorem.html
\item Universality: Voting systems should place no restrictions apart from transitivity on how voters can rank the candidates in an election. There should be no kind of dictatorship presetting that only specific preference orders are acceptable or are not. Every possible collection of transitive preference ballots must yield a transitive societal preference order. 
\item Positive Association of Social and Individual Values: Voting systems should be monotone. 
\item Independence of Irrelevant Alternatives: Voting systems should satisfy the IIA - criterion. 
\item Citizen Sovereignty: Voting systems should not be imposed in any way. I.e., there should never be a pair of candidates in an election such that one of these candidates is preferred over or tied with the other in the resulting societal preference order in defiance of any vote. 
\item Nondictatorship: Voting systems should not be dictatorial. I.e., there should never be a voter such that, for any pair of candidates, if that voter prefers one of the candidates over the other, then society will also have the same preference order regarding these two candidates. 
\end{enumerate} 
The second condition had been called monotonicity so far. The third one is the one we introduced before. The fourth condition is close to neutrality and the fifth is similar to anonymity. 
The first of these conditions is one that we haven't stated before, but which we have been assuming anyways. However, according to this condition a voting system must always output a transitive societal preference order, but some potential voting systems - e.g. sequential pairwise voting - yield cyclic societal preference orders when receiving certain inputs. So one could postulate that only systems which output transitive societal preference orders are called voting systems by definition or one could handle the problem with systems of this kind by excluding inputs that yield preferential orders which are not transitive through restrictions. The latter option then in some way would violate universality. \\
What Kenneth arrow discovered is the following: For an eleciton with three or more candidates, it is impossible for a voting system to satisfy all five of the conditions above. \\
In other words, there is no "perfect" voting system for polls with more than two candidates, at least one sensible condition always will be violated. But there is a great variety of interpretations of Arrow's Theorem, what it really means and how it should be viewed in light of the search for a voting system that is truly fair. 
One of these is Pareto's Unanimity condition, which replaces the conditions of Positive Association of Social and Individual Values and Citizen Sovereignty. It postulates that if there is a pair of candidates in an election such that every voter prefers the same one over the other, then that one should be ranked higher than the other in the resulting societal preference order. 
Since the holding true of Unanimity implies that Positive Association of Social and Individual Values and Citizen Sovereignty also are fulfilled, one can formulate a stronger form aof Arrow's Theorem: \\
For an election with more than two candidates, a voting system can't satisfy Universality, Independence of Irrelevant Alternatives, Nondictatorship and Unanimity all at once. \\
This stronger version is easier to prove. In order to do so, the following Lemma is needed: \\

Assume having an electin with three or more candidates and a voting system that satisfies universality, IIA and unanimity. Suppose that A is a candidate in the election and that every voter ranks A either first or last on their individual preference order (whereby some can rank him first and some last, not all have to rank him the same; furthermore this assumption excludes ties between A and any other candidate). Then the societal preference order the voting system yields, must also rank A either first or last. \\

Proof: If A was neither ranked alone first nor last in the societal preference order, there would have to be some other candidates B and C, so that B was ranked higher than or tied with A, and C was ranked lower than or tied with A in the societal preference order. 
Due to universality, transitivity holds and so B would be ranked higher than C or would be tied with him. \\

Furthermore, if the assumption of the Lemma holds and every voter changed the position of C above B in their personal preference order, the societal preference between A and B as well as A and C wouldn't be affected since A was ranked highest or lowest and thus the relation between A and B as well as A and C wouldn't change. Due to unanimity, we know that C would also be ranked higher than B in the societal preference order then. 



% maybe add instant runoff (p 48-50) though it violates monotonicity and CWC
% TODO: majority criterion ergänzen

%https://en.wikipedia.org/wiki/Social_Choice_and_Individual_Values

%TODO: May's theorem


%%@Alex
%%
%% plz don't use German in this text (also comments). 
%%It is harder to read for outsiders and also a problem when you want to to grammar correction.
%%
%% And I propose to do line breaks after sentences to make the diff be easier to read.
%% (At the very latest after the dot, possibly every 80 chars, even.)
%% If you do a line break without \\, it will be ignored in the formatted text anyway.




{\color{blue}{
\begin{verbatim}
##### Voting system criteria

https://wiki.electorama.com/wiki/Category:Voting_system_criteria
%%TODO: 
%% move wikipedia links to references


###### Features for classifications of voting systems
* Plurality voting
* Instant-runoff voting

### Popular voting systems
##### First-past-the-post voting/Winner takes it all
- Voters indicate on a ballot the candidate of their choice, and the candidate who receives the most votes wins. 
- If there are at least two positions two be filled, each voter casts (up to) the same number of votes as there are positions to be filled, and those elected are the highest-placed candidates corresponding to that number of positions. 
- (huge downside: it very much encourages tactical voting)
- https://en.wikipedia.org/wiki/First-past-the-post_voting 

##### Majority judgement
- Used to determine a single winner. 
- Voters grade each candidate in one of several ranks, for instance named from "excellent" to "bad", and the candidate with the highest median grade is the winner.
- The system's inventors mathematically proved that the system was the most "strategy-resistant" and therefore somehow immune to tactical voting.
- The algorithm we propose is also based on the median grade. 
https://en.wikipedia.org/wiki/Majority_judgment

##### Approval voting
- Usually used to determine a single winner. 
- Each voter may select any number of candidates. The winner is the candidate who is selected ("approved") most often. 
- Variation: each voter may only select a predetermined number of candidates, otherwise this person's cast votes are invalid. 
- The algorithm we propose also allows each candidate to evaluate an arbitrary number of candidates, but it is far from being some sort of approval voting. 
https://www.electology.org/approval-voting
https://de.wikipedia.org/wiki/Wahl_durch_Zustimmung

##### Closed list voting
- Used when positions are to be filled by members of the running parties. 
- Closed list voting: The voters only can vote for the parties. The elected parties have full decision over who of their members get the positions. 
In praxis, the order in which a party's list candidates get elected can also be predetermined by districts. Then the voting system is called a "local list" system.  

- These systems aren't of any use for our considerations. 
- https://en.wikipedia.org/wiki/Closed_list; https://en.wikipedia.org/wiki/Party-list_proportional_representation

#### Preferential voting/Preference voting: 
May refer to different election systems or groups of election systems. Some authors consider preferentially to be one of the characteristics by which electoral systems can be evaluated and classified. 
Preferential voting may, for example, refer to ranked voting methods/ordinal voting systems, i.e. all election methods that involve ranking candidates in order of preference in a hierarchy on the ordinal scale, the most important ones being instant-runoff voting, single transferable vote and the Borda count. 
instant-runoff voting
range voting/score voting
open list
bucklin voting

##### Score voting/rate voting 
- Used to determine a single winner. 
- Voters rate candidates on a scale. The candidate with the highest rating wins.  
- Variations of score voting can use a score-style ballot to elect multiple candidates simultaneously.

%%TODOTODOTODO: 
%%comparison to majority judgement, approval voting and bucklin voting (similarities/differences)

##### Instant-runoff voting 
Used in single-seat elections with more than two candidates 
Voters rank all of the candidates in their personal order of preference
The candidate with the fewest first-choice-votes is eliminated. If there is more than one candidate with the fewest first-choice-votes, the second-choice-votes of these candidates are taken into consideration and so on in order to eliminate only one candidate per voting round. 
In the last voting round there are only two candidates left. The one who gets a majority of first-choice-votes wins the election. 
Variations: There are a few variations of Instant-runoff voting. For example, a candidate could be considered the winner as soon as he/she has a majority of first-choice-votes even though there are more than two candidates left. 
Downside for our purpose: well known (old) dApps are far too hard to be taken over

#### Open list voting
Voters have at least some influence on the order in which a party's candidates are elected. (Example: The candidates are the party's members. Each party has a rank of their members. If a party's candidate reaches a certain quota of the votes, he gets the position even if he's not at the top of the party's list.) 
Open list describes a certain family of voting systems for elections in which multiple candidates are elected through allocations to an electoral listing. What this family has in common is that voters have at least some influence on the order in which a party's candidates are elected. 
In "relatively closed" list systems, a candidate must get a full quota of votes to win a seat. There are various quotas that could be considered for this purpose. 
In "more open" list systems, that quota is so low that it's possible that more of a party's candidates achieve the quota than the total number of seats won by the party. Therefore it should be constituted in advance whether the party's list ranking or the total number of each candidates' votes takes precedence if this happens.
In the "most open" list system, the total number of votes each candidate has received utterly determines the election result. 
In a "free list" system/panachage electors even have more power over which candidates are elected than in the systems above, because each elector is given as many votes as there are seats to be filled and is allowed to cast more than one vote to the very same candidate.
https://en.wikipedia.org/wiki/Open_list; https://en.wikipedia.org/wiki/Party-list_proportional_representation

##### Bucklin voting/The Grand Junction System: 
- Usually used to determine a single winner. 
- Each voter ranks the candidates in ascending order, the first one being the favourite and the last one being the worst candidate. 
- To evaluate the voting's, at first the prime rank is considered. Each candidate gains a score as high as the number of voters that ranked him first. If a candidate's total votes reach a majority (i.e. more than half of the number of voters), this candidate is the winner of the election. If not, the second rank is taken into consideration. 
- The number of votes each candidate received at the second rank is added to the number of votes from the first rank. If a candidates number of votes reaches the majority (i.e. more than half of the number of voters), this candidate wins the election. Otherwise the procedure continues until a candidate with a majority vote is found. The winner then is the candidate with the most votes accumulated.
https://en.wikipedia.org/wiki/Bucklin_voting 
https://de.wikipedia.org/wiki/Bucklin-Wahl
https://www.youtube.com/watch?v=CkIYZsJAvNQ

#### Ranked voting/Ordinal voting systems 
Ranked voting describes certain voting systems in which voters rank outcomes in a hierarchy on the ordinal scale (ordinal voting systems). 


#### cardinal voting systems


\end{verbatim}
}}%% end color

 
{\color{green}{\begin{verbatim}
https://de.wikipedia.org/wiki/Majority_Judgment
https://www.electology.org/score-voting
https://en.wikipedia.org/wiki/Score_voting 

https://en.wikipedia.org/wiki/Instant-runoff_voting (todo)
https://en.wikipedia.org/wiki/Open_list
https://en.wikipedia.org/wiki/Preferential_voting (todo)

https://en.wikipedia.org/wiki/Ranked_voting (todo)
%%TODO: 
%% move wikipedia links to references
\end{verbatim}}}%% end color

