
\section{Propose an algorithm}

{\color{red}{
TODO: explain the below in detail and put it in context
}}

\begin{lstlisting}[language=Python]

rankings = {
    'Alice': 
    {
        'B+': ['cherry', 'orange'], 
        'D': ['banana', 'apple', 'kiwi']
    },
    
    'Bob': 
    {
        'B+': ['orange'], 
        'D+': ['kiwi', 'pear'], 
        'D': ['apple', 'banana']
    },
    
    'Carl': {
        'A+': ['pear', 'apple'], 
        'A': ['cherry'], 
        'C+': ['peach'], 
        'C': ['orange', 'kiwi'], 
        'F': ['banana']
    }
}
    
## collect data
grade_bases = to_rating_dict(GRADES)

fruit_user_ratings = {fruit: {} for fruit in FRUITS}
user_fruit_ratings = {user : {} for user  in USERS }

for user in USERS:
    print('\nuser: {} '.format(user) + 40*'-')
    user_ranking = rankings[user]
    for grade in GRADES:
        if grade in user_ranking.keys():
            gb = grade_bases[grade]
            l = user_ranking[grade]
            r = to_rating_dict(l)
            for fruit, step_rating in r.items():
                fr = gb-(1-step_rating)/len(GRADES)
                fruit_user_ratings[fruit][user] = fr
                user_fruit_ratings[user][fruit] = fr

## compute bracketed means
fruit_user_ratings_means = {fruit: mean(rating_dict.values()) for fruit, rating_dict in fruit_user_ratings.items()}
fruit_user_ratings_means_sorted = sorted(fruit_user_ratings_means.items(), key=operator.itemgetter(1))

\end{lstlisting}

{\color{red}{
TODO: make the above code prettier
}}
