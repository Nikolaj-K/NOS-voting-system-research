%TODO: specifiy, for which elections (2 candidates etc) the following holds

One way to measure the fairness of a given voting system is to determine some properties that the system should satisfy. 
Commonly, the three properties that a voting system should fulfil in order to be fair are anonymity, neutrality and monotonicity. For two-candidate elections they are defined as follows: \\
Anonymity: A voting system is said to be anonymous if it treats all voters equally. I.e. if any two voters trade ballots, this shouldn't change the election's outcome. \\
Concerning the dApp-voting-system, this property could be argued. For example, it could be sensible to give users who have a high reputation, which indicates their knowledge, or users who hold a large stake and therefore are likely to want the best for the platform, more voting power than others. Certainly it would establish an unwanted great inequality between users if the relation between voting power and reputation or stake was a linear one. We discuss this topic further in \\%\ref{label} S. 19: Majority rule is the voting system that elects the candidate who receives more than half of the votes, if such a candidate exists. If there is no such candidate, then majority rule results in a tie, with no winner elected. 
Neutrality: A voting system is said to be neutral if it treats all candidates equally. I.e. if every voter switched their vote from one candidate to another, the outcome should change accordingly. \\
Monotonicity: A voting system is said to be monotone if it is impossible for a candidate to change from winning to losing by gaining additional votes and to change from losing to winning by losing votes without gaining others. \\
The so-called May's Theorem states that majority rule %TODO: see \ref{} 
is the only voting system for two-candidate-elections that is anonymous, neutral, and monotone, and that avoids the possibility of ties. However, majority rule is no option for the dApp-voting-system because there will be more than two candidates and anonymity is no property wanted. \\

%\cite{Handsonapproach} The Mathematics of Voting and Elections: A Hands-On Approach. Chapter One; maybe TODO: quota systems

Some generalisation of majority rule, which majority rule is a special case of, is plurality method. It is the voting system that elects the candidate who receives the largest number of votes, even if that number is less than half of the total number of votes cast. Plurality method only results in a tie, when two or more candidates receive the same number of votes and more than the number of votes received by any of the other candidates. \\
Concerning the dApp-voting-system, the possibility of ties is no downside since there's no need to determine a single winner.\\ %S. 19
Rather, what is needed is a system that leads to some sort of preference order of all dApps. Such preference order produced by the voting system is called "societal preference order" since it can be thought of as the ranking of the candidates that, according to the voting system being used, best reflects the voters' will. \\
There are various system that can be used to determine the societal preference order. One property of such systems that might sound sensible at first, is the following: \\
A voting system is said to satisfy the majority criterion if whenever a candidate is ranked first by a majority of the voters, that candidate will also be ranked first in the corresponding societal preference order. We give an example for why this would be no legitimate property for the dApp-voting-system in a moment.\\
Some voting system that does not fulfil the majority criterion, is the so-called Borda count which uses a point system to determine overall rankings and is often used in collegiate sports polls for example. In an election with n candidates it works as follows: \\
Firstly, each voter submits a ballot that contains his or her individual preference order of all the candidates. \\
Then points are awarded to each candidate for each ballot cast, according to the following rule: \\
A m-place ranking is worth n-m points (where $1\leq m \leq n$). In other words, a first-place ranking is worth n-1 points, a second-place ranking is worth n-2 points and so on. 
Finally the candidate whose total number of points from all of the ballots is the largest is declared the winner and the corresponding societal preference order is determined by the number of points each candidate has got from largest to smallest. If there is more than one candidate with the largest number of points, a tie occurs. Also some sort of tie occurs in the societal preference order whenever candidates receive the same number of total points. They then occupy consecutive indistinguiable positions in the preference order. \\
Intuitively, Borda count also seems to be quite fair. However, it violates the majority criterion. For the purpose of the dApp-voting-system we consider using some sort of variation of the Borda count more appropriate than sticking to majority criterion. Imagine an extreme situation where a small majority of voters (i.e. users) ranks a specific dApp first place but all of the other voters rank it last on their personal preference. Obviously it would make no sense to declare this dApp the winner and rank it first in a societal preference order. Determining it's rank according to the Borda count seems much more legitimate. So the dApp-voting-system will not fulfil the majority criterion and will not use the plurality method, but rather use some sort of Borda count in order to lead to a sensible societal preference order. \\
Taking account of personal and societal preference orders, for elections with more than two candidates, the three properties of fair voting systems have to be redefined: \\
Anonymity: A voting system is said to be anonymous if it treats all voters equally. I.e. if any two voters traded their personal preference orders, the outcome of the resulting societal preference order should not change. \\
As mentioned before, this is no property we want the dApp-voting-system to fulfil, see \\ %\ref{}. 
Neutrality: A voting system is said to be neutral if it treats all candidates equally. I.e. if every voter switched the positions of two specific candidates on their personal preference orders, the positions of these two candidates in the resulting societal preference order would be switched accordingly. \\
Monotonicity: A voting system is said to be monotone if it is impossible for a candidate to go from winning to losing or to experience a decrease in rank on the resulting societal preference order whenever changes in favour of that candidate, but no changes in disadvantage of that candidate, occur on individual preference ballots. 
%großes Buch wo alles drinsteht, Chapter Two





%TODO: May's theorem
%TODO: Arrow's theorem http://pi.math.cornell.edu/~mec/Summer2008/anema/maystheorem.html

