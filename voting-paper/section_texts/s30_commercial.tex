
\section{On commercially used ranking systems today}

\subsection{Retailer sector} %%%%%%%%%% %%%%%%%%%% %%%%%%%%%%
Product reviews not only provide valuable information for {\emph{buyers}} before a purchase.
It is also a means to express their experiences and - in this way - socially interact with other users. 
For the owners of the platform, user created content is essentially free value which not only 
evaluates and thereby ranks the product at hand, but also gives a means for targeted advertisement. 

This is but one aspect of professional modern evaluation systems.
In their article on {\emph{Amazon.com, Inc.}}, W. Mitchell lists the most important factors of automatized ranking of 
the leading electronic commerce and cloud computing company [\href{https://startupbros.com/rank-amazon/}{rank-amazon}]: %% move link to bibtex

{\color{green}{
	TODO: redo this list, make many of those items to topics in themselves. If you keep the list as such, write it down more concisely.
}} %% end color

\begin{itemize}
	%% 1.
   \item Conversion Rate Factors:
   \begin{enumerate}
		\item Sales Rank
		\item Customer Reviews
		\item Answered Questions
		\item Image Size and Quality
		\item Price
		\item Parent-Child Products
		\item Time on Page and Bounce Rate
		\item Product Listing Completeness
   \end{enumerate}
	%% 2.
   \item Relevancy Factors:
   \begin{enumerate}
		\item Sales Rank
		\item Customer Reviews
		\item Answered Questions
		\item Image Size and Quality
		\item Price
		\item Parent-Child Products
		\item Time on Page and Bounce Rate
		\item Product Listing Completeness
   \end{enumerate}
	%% 3.
   \item Customer Satisfaction and Retention Factors:
   \begin{enumerate}
		\item Negative Seller Feedback
		\item Order Processing Speed
		\item In-Stock Rate
		\item Perfect Order Percentage (POP)
		\item Order Defect Rate (ODR)
		\item Exit Rate
		\item Packaging Options
   \end{enumerate}
\end{itemize}

Furthermore, the user rating interface of the e-commerce corporation {\emph{eBay Inc.}} is discussed, see {ebay.de}\\ 
%% move link to bibtex
%[https://pages.ebay.de/help/feedback/howitworks.html], 
%[https://verkaeuferportal.ebay.de/verkaeufer-news/2016-fruehling/produktbewertungen-rezensionen], 
%[https://pages.ebay.de/help/feedback/questions/leave.html]:

\begin{itemize}
	%% 1.
   \item Evaluation of sellers:
   \begin{itemize}
		\item Standard evaluation, given by verified buyers:
   			\begin{enumerate}
			\item positive vote: + 1 point
			\item neutral vote: 0 points
			\item negative vote: -1 point
			\item one vote per buyer per week (Mon- Sun) is counted 
			\item 13 different levels of rating of the sellers, symbolized by differently coloured stars
  			\end{enumerate}
		\item Detailed evaluation, may be given after the standard evaluation: 
   			\begin{enumerate}
			\item 1-5 stars (voting points) for each of 4 categories (article, communication, sender time, shipping costs) possible
			\item independent from the standard evaluation, doesn't affect it
			\item one rating per purchase possible
			\item is are shown only if there are at least 10 detailed evaluations
  			\end{enumerate}
   \end{itemize}
	%% 2.
   \item Evaluation of buyers:
   \begin{itemize}
		\item Buyers can be evaluated by the sellers, but only positive votes are possible.
		\item Evaluations can be edited if both parties do agree.
   \end{itemize}
	%% 3.
   \item Evaluation of products: 
   			\begin{enumerate}
				\item 1-5 stars (5 being the best) 
				\item In addition, there are 3 product-specific questions to answer (yes/no) 
				\item The average of the stars-rating and the percentage of positive answers to the questions are shown on the product
				\item Also, people can write reviews; reviews can be given a positive or negative vote or can be reported
  			\end{enumerate}
\end{itemize}




\subsection{Information sector} %%%%%%%%%% %%%%%%%%%% %%%%%%%%%%
\subsubsection{Q and A platforms}
   
\begin{itemize}
	\item StackExchange
	\item Quora
	\item Reddit
\end{itemize}

\subsubsection{News platforms}

\subsection{Entertainment sector} %%%%%%%%%% %%%%%%%%%% %%%%%%%%%%

{\color{blue}{
VotingPlugin (plugin for Minecraft) allows one to give his players rewards by voting for his servers. \\
Types of rewards: 
\begin{itemize}
	\item for votes for one site
	\item for voting on all of some specified sites
	\item for the first vote
	\item cummulative reward (vote x amount of times to be rewarded (per day/week)) 
	\item for voting x number of times in a row
	\item for x amount of global votes
\end{itemize}
(source: https://www.spigotmc.org/resources/votingplugin.15358/)
}} %% end color

\subsection{Blockchain sector} %%%%%%%%%% %%%%%%%%%% %%%%%%%%%%

{\color{blue}{
\begin{itemize}
    \item Lisk voting, earnlisk.com
    \item augur "reporting": 50% ROI
    \item Gnosis
    \item reward voting 7.0
    \item openbazaar
    \item repu-coin
    \item odem.io
    \item riskbazaar
    \item drep.org
    \item stackexchange
\end{itemize}
}} %% end color
    
\subsubsection{Lisk}
{\color{blue}{
	\begin{itemize}
		\item delegate proof of stake --> one earns lisk by voting for delegates who share their rewards with their voters (max. number of 		  votes: 101) 
		\item 4 batches á max. 33 votes (max. 101 votes at altogether) to participate; 
		\item to participate at a batch, one has to pay 1 lisk, which has to be in the lisk-wallet
		\item (Open question: what happens, if voted delegators don't win --> is the paid lisk just lost?) (source: https://earnlisk.com/)	
	\end{itemize}
}} %% end color

\subsection{User experience} %%%%%%%%%% %%%%%%%%%% %%%%%%%%%%

(reflect on the above and what we want)%% remove comment


